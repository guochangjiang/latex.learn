%-*- coding: UTF-8 -*-
% gougu.tex
% 勾股定理
\documentclass[UTF8]{ctexart}

\title{杂谈勾股定理}
\author{张三}
\date{\today}

\bibliographystyle{plain}

\begin{document}

\maketitle
\tableofcontents
\section{勾股定理在古代}
西方称勾股定理为毕达哥拉斯定理,将勾股定理的发现归功于公元前 6 世纪的毕达哥拉斯学派[1]。该学派得到了一个法则,可以求出可排成直角三角形三边的三元数组。毕达哥拉斯学派没有书面著作,该定理的严格表述和证明则见于欧几里得《几何原本》的命题47:“直角三角形斜边上的正方形等于两直角边上的两个正方形之和。”证明是用面积做的。

我国《周髀算经》载商高(约公元前 12 世纪)答周公问:
勾广三,股修四,径隅五。
又载陈子(约公元前 7-6 世纪)答荣方问:
若求邪至日者,以日下为勾,日高为股,勾股各自乘,并而开方除之,得邪至日。
都较古希腊更早。后者已经明确道出勾股定理的一般形式。图 1 是我国古代对勾股定理的一种证明[2]。

\section{勾股定理的近代形式}
\bibliography{math}

\end{document}
